%%novalidate

\beginsong{La copla del rondador}[by={Julio Nuel}]

    \usebox{\A} \usebox{\EVII} \usebox{\Bm}

    %%novalidate

\beginverse
    \[A] \[E7] \[A]
\endverse

\setlength\parindent{0pt}
\begin{music}
    \instrumentnumber{1}
    \nobarnumbers
    \TAB1
    \setlines1{6}
    \startpiece
        \Notes\hsk\STr42\en
        \Notes\Str40\en
        \Notes\Str52\en
        \Notes\Str64\en
        \Notes\str60\en
  \endpiece
\end{music}

    \beginverse
        Aquí está la Tuna que con su ale\[A]gría
        Recorre las calles con una can\[E7]ción
        Y con su ban\[Bm]dera \[E7] y con su ale\[Bm]gría
        Alegran la \[E7]vida de la pobla\[A]ción
    \endverse

    \beginverse
        Somos estudiantes muchachos de \[A]honra
        De buenas palabras y gran cora\[E7]zón
        Somos trova\[Bm]dores \[E7] que llevan sus \[Bm]coplas
        para las mu\[E7]chachas de nuestra re\[A]gión
    \endverse

    %%novalidate

\beginverse
    \[A] \[E7] \[A]
\endverse

\setlength\parindent{0pt}
\begin{music}
    \instrumentnumber{1}
    \nobarnumbers
    \TAB1
    \setlines1{6}
    \startpiece
        \Notes\hsk\STr42\en
        \Notes\Str40\en
        \Notes\Str52\en
        \Notes\Str64\en
        \Notes\str60\en
  \endpiece
\end{music}

    \beginverse
        Canta una copla la Tuna
        \[A]La copla del ronda\[E7]dor
        Canta una copla la \[A]Tuna
        Para que salgas mo\[E7]rena a ver \[Bm]a tu \[E7] ronda\[A]dor
        Para que salgas mo\[E7]rena a ver \[Bm]a tu \[E7] ronda\[A]dor
    \endverse
    
    %%novalidate

\beginverse
    \[A] \[E7] \[A]
\endverse

\setlength\parindent{0pt}
\begin{music}
    \instrumentnumber{1}
    \nobarnumbers
    \TAB1
    \setlines1{6}
    \startpiece
        \Notes\hsk\STr42\en
        \Notes\Str40\en
        \Notes\Str52\en
        \Notes\Str64\en
        \Notes\str60\en
  \endpiece
\end{music}

    \beginverse
        Canta otra copla la Tuna
        \[A]La copla del ronda\[E7]dor
        Canta otra copla la \[A]Tuna
        Para que salgas mo\[E7]rena a ver \[Bm]a tu \[E7] ronda\[A]dor
        Para que salgas mo\[E7]rena a ver \[Bm]a tu \[E7] ronda\[A]dor \[E-A]
    \endverse

\endsong

\scleardpage